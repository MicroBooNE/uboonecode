\documentclass[12pt]{article}
\usepackage{epsfig,graphics,epsf,subfigure}
%%%\usepackage{amsmath}
%%%\usepackage{footnote}
%%%\usepackage{amssymb}
%%%\usepackage{url}
%%%\usepackage{array}
\setcounter{tocdepth}{2}

\parindent0cm
\parskip 0.5em
\renewcommand{\arraystretch}{1.2}
\renewcommand{\topfraction}{0.9999}
\renewcommand{\bottomfraction}{0.9999}
\renewcommand{\textfraction}{0.0}
\hoffset-0.5cm
\voffset-2.0cm
\textwidth15.0truecm
\textheight23.0truecm
\def\begi{\begin{itemize}}
\def\endi{\end{itemize}}


\begin{document}
\pagestyle{plain}
\pagenumbering{none}
\begin{flushright}D\O~Note 5323 \\
January 14, 2007
\end{flushright}
\bigskip
\begin{center}
{ \Huge \textbf{GMBrowser User Manual }}\\
\bigskip
Elliott Cheu, {\it University of Arizona, Tucson, Arizona} \\
Harrison Prosper, {\it Florida State University, Tallahasee, Florida}
\end{center}
\begin{abstract}
This note describes the usage of the gmbrowser histogram browser utility. 
gmbrowser is a package which can be used for monitoring histograms in
real time. It is currently used by many of the online examine processes
to display various quantities used in monitoring the health of the 
D\O\ detector and its data.
\end{abstract}
\vspace*{0.5cm}
\tableofcontents
% if table of contents is one page only next page should be empty
\thispagestyle{empty}
\mbox{}
%%%\newpage
\clearpage
\pagenumbering{arabic}
\setcounter{page}{1}

\section[Introduction]{Introduction}
The gmbrowser process uses a straight-forward configuration file 
to specify pages of plots to display. Many options are available
for determining the layout of each page as well as how each plot
is displayed. Root macros can also be used to generate a given
page. For each plot a reference histogram can also be overlaid
to give the operator an idea of the ideal distribution for a given
plot. Individual plots can also be tagged so that clicking on a plot
will load a new configuration file. This allows one to have the
ability to load expert plots associated with a given plot.

The program reads .root files
that can be located on locally mounted disks or on remote systems.
Remote files are accessed via the rootd protocol supported by root.
More information about rootd can be found at http://root.cern.ch.
The gmbrowser program can connect to .root files that are simultaneously
being written to. This allows for a very flexible means of developing
a monitoring system. Also, multiple input files can be specified which
allows the gmbrowser to add together similar files from multiple
processes.


\section[Configuration Files]{Configuration Files}
The behavior of gmbrowser is driven by a relatively simple 
configuration file. An example of such a file is shown in 
section~\ref{sec:config}.  The main syntax of the 
gmbrowser configuration file is a directive followed by
a value. All gmbrowser directives end in a colon. 
Comments can occur anywhere on a line and begin with a 
pound sign (\#). Lines can be up to 255 characters long and
blank lines are allowed.

The configuration file can be
broken into three main parts: global directives, page directives
and histogram definitions. The global directives specify 
the names of the data and reference histogram files, and
the general behavior of gmbrowser. 
Table~\ref{tab:global}\ lists the global configuration
directives.

The second section of each configuration file specifies 
how each page of plots should look. This section is used
to indicate how many plots should be displayed on each
page and can also specify a macro to run to display a 
given page. 
Table~\ref{tab:page}\ lists the page specific directives.
Each page must have the Page.Title: directive and either
a Page.Division: directive or a Page.Macro.File: (and Page.Macro.Function:)
directive.

The third section of a configuration file contains the list
of histograms to display and their respective options.
In the case of a macro function, this section is not needed.
The number of histograms defined cannot exceed the number
specified in the Page.Division: directive. However, the
number of histograms defined can be fewer than the Page.Division:
directive. The format of the histogram definition line is
as follows:

  \qquad histogram name $|$  histogram title $[|$ options$]$

The histogram name is the name of the histogram
in the .root file. The name of the histogram can include the
subdirectory name if there are subdirectories defined in the
.root file. The histogram
title can be anything you choose, and can include white space. 
Options are used to customize the properties of each plot.
By default each plot is displayed with linear scales, a
plot title and the histogram statistics. Also, a reference
histogram (if it exists) is overlaid on the histogram.
Table~\ref{tab:options}\ lists the available histogram options.

\begin{table}
\begin{tabular}{llp{3.5in}}
\hline\hline
Directive & Values & Usage \\ \hline
  Root.File:     & File Name & Name of the .root file to be read by 
                               gmbrowser.\\
  Root.File.Ref: & File Name & Name of .root file containing reference 
                               histograms.\\
  Root.File.Add: & File Name & Name of file to sum together with the data
                               histogram.\\
  WWW.Dir:       & Path Name & Name of directory to store .gif files for 
      displaying on web server.\\
  Load.Dir:      & Path Name & Directory containing the configuration files.
                               This is only necessary if you want to be 
                               able to read in a new configuration file 
                               while gmbrowser is running.\\
  Save.Dir:      & Path Name & Directory where .jpg, .ps and .eps files 
                               are written.\\
  Update.On:     & 0 or 1    & Turns on(1) or off(0) the automatic 
                               updating (i.e. rereading of the .root files). 
                               This feature is useful if the input .root file
                               is being updated and you want to see an updated 
                               plot. By default Updating is turned off when 
                               gmbrowser starts.\\
  Debug.On:      & 0 or 1    & Turns on(1) or off(0) debugging statements. 
                               This option prints out a large amount of output.
                               So, only use it if you are having a problem 
                               with gmbrowser. \\
  Cycle.On:      & 0 or 1    & Turns on(1) or off(0) the cycling through each
                               defined page of histograms. By default, cycling 
                               is turned off when gmbrowser is started. \\
  Update.Period: & Integer   & Number of seconds between updates if updating
                               is enabled.\\
  Cycle.Period:  & Integer   & Number of seconds each page is displayed when
                               cycling is enabled.\\
  Cycle.Plots:   & 0 or 1    & Turn off or on the generation of plots whenever
                               the cycle option is chosen. This will generate
                               .gif plots in the WWW.Dir and .ps plots in the 
                               Save.Dir areas. Default is on.\\
  TotalEvent.Hist: & Histogram name & Name of a histogram used to determine the
                                      total number of events contained in the
                                      roottuple. gmbrowser gets the number of 
                                      entries from this histogram and displays
                                      it at the bottom of the canvas.\\
  RunNumber.Hist: & Histogram name & Name of histogram used to pass the
                                     run number. This is a 1-D histogram with
                                     a single bin. The area of the bin 
                                     corresponds to the run number. \\
\hline
\end{tabular}
\caption{List of Global Directives for Configuration Files}
\label{tab:global}
\end{table}


\begin{table}
\begin{tabular}{llp{3.in}}
\hline\hline
Directive & Values & Usage \\ \hline
Page.Title: &  Page Name & Title for a given page of histograms.\\
Page.HelpFile: & File Name & Name of file containing a description of the 
                             plots on this page. This is just a text file 
                             which should contain useful information about 
                             the plots.\\
Page.Macro.File: & File Name & 
        Name of a file containing macros. If a macro is defined, then
        one should not use the Page.Division and histogram definition
        directives. The file name should include the path to the macro file.
        It can
        either be the absolute path, or the relative path from where
        the executable is run. 
        An example macro can be found in the macros directory.\\
Page.Macro.Func: & Function Name & 
        Name of the macro routine to be executed. This directive should
        follow the Page.Macro.File directive. The name should just be the name.
        i.e. myfunc. Do NOT include the parentheses (i.e. myfunc()). The 
        definition of myfunc should include the canvas, and pointers to the 
        data and reference histograms. \\
Page.Division: & $\langle$nx$\rangle$ $\langle$ny$\rangle$ &
       Division of histogram page (i.e. nx by ny).
       Useful divisions are (1 1), (2 2) (2 3).
       Following the Page.Division command are the list of
        histograms for this page. Be sure to have no more than
        (nx) X (ny) histograms listed.\\
Page.ShowRunNumber: & 0 or 1 &
       Turns off/on option to show run number in given page.\\
\hline
\end{tabular}
\caption{List of Page Specific Directives}
\label{tab:page}
\end{table}

\begin{table}
\begin{tabular}{lp{3.5in}}
\hline\hline
Option & Result \\ \hline
 logx             & Use logarithmic x axis\\
 logy             & Use logarithmic y axis\\
 logz             & Use logarithmic z axis\\
 limits(ymin:ymax) & Fix the min and max for the y axis.
                         Be sure there are no spaces 
                         i.e. limits(0:500).
                         If the maximum value is set to zero, then
                         the maximum will float while the minimum
                         will be set to the ymin value.\\
 range(xmin:xmax)      & Sets limits for x axis.\\
 rebin(nbin)           & Rebins with nbin bins.\\
 scale\_ymin(scale)    & sets the minimum value of the y axis to be
                         scale\_ymin $\times$ y\_max $\times -1$. This
                         is so that one can see the $y=0$ values more easily.\\
 stats(mode)           & Change appearance of statistics box
                         mode has form used in
                         gStyle-$>$SetOptStat(1111).\\
 lwid(width)           & Set the histogram line width (in pixels)
                         minimum number is 1.\\
 config($\langle$file.cfg$\rangle$)  & Associates a new configuration file with
                         the named histogram. When the user double
                         clicks on the displayed histogram, the new
                         configuration file is read in and gmbrowser
                         pages are updated. To return to the previous
                         configuration file, click on the ``Previous
                         Config" button.\\
 noscale               & Doesn't rescale the reference histogram to the
                         displayed histogram.\\
 noKolfit              & Don't perform the Kolmogorov test between the
                         reference and displayed histogram.\\
 noRef                 & turn off displaying of reference histogram.\\
 noRef2                 & turn off displaying of the second reference 
                          histogram.\\
 lego, surf, etc.      & Any 2D plot supported by the root command SetOption(option) \\
 Any root histogram option & i.e. H B C E E0 E1 E2 E3 E4 L P (for 1D plots)
                             BOX COL etc. (for 2D plots)\\
\hline
\end{tabular}
\caption{List of Histogram Options}
\label{tab:options}
\end{table}

\subsection[Specifying Root Files]{Specifying Root Files}
The directives Root.File: and Root.File.Ref: are used to 
specify the path to the appropriate .root files. One can use
the root:  syntax in the file name to specify that
the file will be accessed remotely. See http://root.cern.ch/root/NetFile.html
for more information.

One can specify more than one Root.File: directive in a configuration
file. This allows one to use a single gmbrowser process to look at
histograms from many different processes (and servers). Just put the
Root.File: (and Root.File.Ref:) directive ahead of the page definition.
Any subsequent page will attempt to read the histograms from
the appropriate file. There is no limitation on the number of 
files that can be accessed by gmbrowser.

\subsection[Reference Histograms]{Reference Histograms}
The Root.File.Ref: and Root.File.Ref2: directives allow one to 
specify a .root file containing a reference histograms. These
options permit one to indicate an ideal shape for each plot.
The .Ref2 directive allows one to show a second reference plot.
At the moment, if the first reference histogram is not defined,
the second reference histogram will not be displayed.
For example if you have a high intensity and low intensity reference,
they can both be shown on the same plot. The reference histograms
can be turned off on a plot-by-plot basis. For each plot,
just use the noRef or noRef2 histogram option. By default,
a Kolmogorov test is performed between the data plot and the
reference plot. The result of the fit is displayed on each
plot. To turn off this behavior, use the noKolfit histogram
option.

\subsection[Summing Histogram Files]{Summing Histogram Files}
The directive Root.File.Add: allows one to sum together multiple
histogram files. This only applies to the data file and not the
reference file. Adding one or more of these directives after
the Root.File: directive will cause gmbrowser to sum the histograms
from each of the specified files before displaying the appropriate
page. 

\subsection[Macros]{Macros}
If one wants to utilize the full power of root, one can write
a macro that is executed whenever the page is displayed.
If a macro is desired, then one does not specify any
histogram names for that given page. Pages of macros can
be interspersed with pages of defined histograms. The
Page.Macro.File: directive specifies the name of the
text file containing the macro. The Page.Macro.Func: directive
then specifies the name of the macro function to execute for
the given page. There can be more than one macro function 
defined in a given macro file. The function has to be
specified in the following way in the macro file:
\begin{verbatim}
        void myfunc(TCanvas *c1, TFile *data, TFile *ref)
\end{verbatim}
The first argument is used to pass the pointer to the
canvas where the plot will be drawn. The next two 
arguments specify the files for the data and reference
histograms. Examples of macros for gmbrowser can be seen
in the macros directory in cvs.

\subsection[Expert Histograms]{Expert Histograms}
A feature that might be useful for a shifter is the ability 
to associate more detailed plots with a given plot. This 
is implemented by using the ``config'' histogram option.
What this feature does is to load a new configuration file
when a specific histogram is clicked on. So, if a user
wants to find out the details of a given plot, he/she would
click on that plot. This would have the effect of loading
a new configuration file, and therefore show a new set of
related plots. There is a ``Previous Config'' button on
gmbrowser which reloads the previous configuration, thereby
going back to the previous set of plots.

\subsection[Plot Descriptions]{Plot Descriptions}
By using the Page.HelpFile: directive, one can specify a text file
describing the plots on a given page. If this directive is
used, then a user can push the ``Plot Info'' button 
on gmbrowser, and the text file will be displayed in a separate
window.

\section[Building gmbrowser]{Building gmbrowser}
The code for gmbrowser can be obtained from cvs. For example:

  \qquad cvs checkout gmbrowser

To build the executable you need to have \$ROOTSYS defined
(i.e. setup root), and \$ROOTSYS/lib must be included in
\$LD\_LIBRARY\_PATH.

Then, typing make in the gmbrowser top dirctory should properly build
gmbrowser. Note that because of the way that gmbrowser implements
macro files, you might get an error like ``GMBowser inherits from
TObject but does not have its own ClassDef''. This error is not
fatal and invoking make a second time usually fixes things.

\section[Invoking and Running gmbrowser]{Invoking and Running gmbrowser}

  Before running gmbrowser, you will need to create a few configuration
  files and set up some environment variables.
  \begi

  \item \$GM\_ICON\_DIR
    points to a directory containing icon files used by gmbrowser.
    If this environment variable is not set, a warning message is
    written by gmbrowser and a few buttons will not be available.
    However, you will be able to run gmbrowser even if this variable
    is not set. The icons are included in the gmbrowser cvs area.

  \item Create a configuration file (i.e. TrigEx.cfg).
    See below.

  \item \$HOME/.GMBrowser
    Contains some start up preferences. This file is not necessary
    but will set defaults if it exists. The available .GMBrowser
    commands are listed in section \ref{sec:GMBrowser}.
\endi

gmbrowser can be started using the following syntax:

\qquad gmbrowser $\langle$Configuration File$\rangle$

Once the process is running an x-window display should appear
on the user's workstation. There are four buttons on the top
right of the display.
\begi
  \item Plot Info: This button brings up a text file (if it exists)
        describing the plots on the page.
  \item Prev Config: This button reloads the previous configuration file used.
        It is mostly useful when expert plots have been defined.
  \item Update: This button causes gmbrowser to reload the plots on the
        page after a predefined time limit.
  \item Cycle: This causes gmbrowser to cycle through all defined pages 
        of histograms. When the browser is cycling, it automatically
        generates a .ps file of all current plots.
\endi
The File button at the top right allows one to make .gif and .ps copies
of the current page of plots. One can also make a .ps file of all of the
pages of hisgtograms, though this takes a bit of time since gmbrowser has
to cycle through all defined histogram pages.

Printing can be done via the File button. One has the option to print
the currently displayed page, or all available pages. The default
printer can be set in the .GMBrowser file.

The menu bar on the left hand side of the browser displays the list of
all defined histogram pages. The current page is highlighted. If one
wants to display a specific histogram page, one can just click on the
appropriate page name.

If expert histograms are defined, clicking on a specific histogram
plot will load a new configuration file that points to the defined
expert plots.

Exiting from gmbrowser can be done through the File menu on the
top left.

\section{Sample Configuration File}
\label{sec:config}
{
  \small
  \begin{verbatim}

#-------------------------------------------------------------------------
# File:		TrigEx.cfg
# Description:	Configuration file for Global Monitor Histogram Browser
#               for Trigger Examine
# Created:	08-Oct-2002 Pushpa Bhat
#		12-Oct-2002 HBP move printer config to $HOME/.GMBrowser
#				Add WWW.Dir etc.
#               23-Nov-2002 Pushpa  Add MET plots
#               10-Dec-2002 ECC - Remove muon plots and add EM_HI_F0 plots.
# 
#-------------------------------------------------------------------------
# !!!!!!!!! Important: Lines can not be longer than 255 characters.
# Configuration files
Load.Dir:	$GM_SCRIPTS_DIR

# Plots etc.
Save.Dir:	.
WWW.Dir:	/mnt/www/htdocs/groups/gm/TrigEx

Root.File:	http://kaon1.physics.arizona.edu/~elliott/TrigEx.root
Root.File.Ref:	http://kaon1.physics.arizona.edu/~elliott/TrigEx_ref.root 

Update.Period:	3	# seconds
Cycle.Period:	10	# seconds

Page.Title:    L1 Muon Triggers
Page.Helpfile: $GM_HELP_DIR/L1Muon.hlp
Page.Division: 2 2
	L1Mu_ScintTrigs	| L1 Muon Scintillator Triggers
	L1Mu_WireTrigs  | L1 Muon Wire Triggers
	L1Mu_BOTTrigs  | L1 Muon Beginning of Turn Triggers

Page.Title:    L2 global Jets
Page.Division: 3 2
	h_N_L2jets   | number of L2 jets
	h_Pt_L2jets  | pt of L2 jets
	h_Eta_L2jets | eta of L2 jets
	h_Phi_L2jets | Phi of L2 jets
	h_EtaPhi_L2jets | eta vs. phi of L2 jets

Page.Title:    L2 global EM Objects 1
Page.Division: 3 2
	h_N_L2emobjects | Number of L2 EM objects
	h_Pt_L2emobjects     | L2 EM objects Pt
	h_Eta_L2emobjects    | L2 EM objects Eta
	h_Phi_L2emobjects    | L2 EM objects Phi
	h_EtaPhi_L2emobjects | L2 EM objects Eta vs Phi

Page.Title:    L2 global EM Objects 2
Page.Division: 1 2
	h_Iso_L2emobjects    | L2 EM objects Iso
	h_EMFrac_L2emobjects | L2 EM objects Emfraction

Page.Title:   Macro Test
Page.Macro.File: test.C
Page.Macro.Func: myfunc

  \end{verbatim}
}
\section{Sample .GMBrowser file}
\label{sec:GMBrowser}
This section lists all of the available commands that can
be used in the .GMBrowser file.
{
  \small
  \begin{verbatim}
Created:	        Fri Oct 11 00:37:52 2002
Printer.Command:        flpr -q
Printer:                dab1_hp8000
Canvas.Color:           42
Update:                 yes
Cycle:                  yes
  \end{verbatim}
}

\section{Summary}
The gmbrowser program provides an easy way to connect to 
running processes to view histograms. It is controlled by a 
single configuration file that is read in at run time. gmbrowser
can be used to view processes running locally or remotely and
provides a number of tools that allow one to easily control the 
look and feel of the browser. 

\end{document}





